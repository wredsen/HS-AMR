\chapter{Implementierung}

\section{Missionsplaner}
Ausgangspunkt für die Implementierung ist das gegebene Beispielprogramm. In diesem wurde die Hauptzustandsmaschine mit drei Grundzuständen (DRIVE, INACTIVE und EXIT) vorgegeben. Mit Hilfe von Switch-Case-Abfragen wird nur der Code des entsprechenden Zustandes ausgeführt. Nach der Abfrage wird überprüft, ob in einem neuen Zustand gewechselt werden muss.\\


\noindent Die Grundversion verfügt noch über keine Unterzustandsmaschinen. Es fehlen auch die wesentlichen Zustände für das Ein- und Ausparken. Der getätigte Entwurf des Missionsplaners wurde mit weiter Switch-Case-Abfragen implementiert.

\section{Pfadgenerator}

Zusammen mit dem Controlmodul wurde entschieden, dass im Guidancemodul die Start- und Zielpose, die Geschwindigkeit übergeben wird und zum Schluss der richtige Controlmodus aktiviert wird. Alle Berechnungen werden im Controlmodul getätigt.\\


Damit das Polynom richtig berechnet wird und anschließend dieses abgefahren werden kann, muss zwischen dem lokalem und globalem Koordinatensystem transformiert werden.

\newpage

\begin{lstlisting}[language=java, frame=single]
// transform coordinates
if(this.destination.getHeading() == 0 ) {
	// do nothing
}else if(Math.abs(this.destination.getHeading() - Math.PI/2) < 0.0001) {
	double n = x_local;
	x_local = y_local;
	y_local = -n;
}else if(Math.abs(this.destination.getHeading() - Math.PI) < 0.0001) {
	x_local = -x_local;
	y_local = -y_local;
}
\end{lstlisting}

\noindent und
 
\begin{lstlisting}[language=java, frame=single]
// back transformation
if(this.destination.getHeading() == 0 ) {
	// do nothing
}else if(Math.abs(this.destination.getHeading() - Math.PI/2) < 0.0001) {
	double n = x_local;
	x_local = -y_local;
	y_local = n;
}else if(Math.abs(this.destination.getHeading() - Math.PI) < 0.0001) {
	x_local = -x_local;
	y_local = -y_local;
}
\end{lstlisting}